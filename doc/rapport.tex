\documentclass{article}

\usepackage[utf8]{inputenc}
\usepackage[T1]{fontenc}
\usepackage[top=1.5in, bottom=1.5in, left=1in, right=1in, headsep=0.5in]{geometry}
\usepackage[frenchb]{babel}
\usepackage{array}
\usepackage{fancyhdr}
\usepackage{amssymb}
\usepackage[final]{pdfpages}

\pagestyle{fancy}
\rhead{Rétro-débogueur pour le langage d'assemblage PEP/8 }
\lhead{INF7741 - Groupe 40}


% pour éviter d'avoir à faire des \noindent partout! 
 
\title{ 
	\Large{Université du Québec à Montréal}\\ 
	\vspace{3cm} 
	\huge{Projet de session: Rétro-débogueur pour le langage d'assemblage PEP/8}\\ 
	\vspace{3cm} 
	\Large{Travail présenté à \\M. Étienne Gagnon} \\ 
	\vspace{1.5cm} 
	\Large{Dans le cadre du cours \\INF7741 – Machines Virtuelles} \\ 
	\vspace{1cm} 
	\author{Frédéric Vachon\\VACF30098405\\Philippe Pépos Petitclerc\\PEPP03049109} 
	\date{\vspace{0.5cm} 21 avril 2016} 
	\vfill 
} 
 
\begin{document} 
\maketitle 

\tableofcontents

\thispagestyle{empty} 
\clearpage 

\openup .5em

\section{Introduction}\label{introduction}

Une grande partie du travail d'un développeur est d'effectuer le
débogage des programmes. La plupart du temps, le développeur doit
relancer plusieurs fois l'application avant de trouver la cause du
bogue. Si le développeur avait accès à un débogueur qui lui permettait
d'effectuer une exécution inverse, sa productivité globale augmenterait.
Ce constat est également vrai dans un contexte de sécurité informatique
où un développeur cherche à profiter d'une faille de sécurité dans un
logiciel pour suivre le déploiement de son code injecté dans la mémoire
du programme. C'est pour répondre à ces deux besoins que nous avons
décidé d'élaborer un projet de débogage à reculons. L'aspect sécurité
informatique nécessite que nous utilisions un langage très bas niveau.
Puisque le langage assembleur \emph{PEP/8} est enseigné à l'Université
du Québec à Montréal et que ce langage ne possède qu'un ensemble
d'instruction très simple, il était un candidat idéal pour développer
notre projet dans le cadre d'un cours de maîtrise.

Le document suivant étaye le travail effectué au cours de la session sur
l'élaboration d'un débogueur à reculons pour le langage d'assemblage
\emph{PEP/8}. Il présente les objectifs du projet et les éléments
nécessaires à sa réalisation. Nous faisons ensuite une revue de
littérature et les articles qui nous ont permis de dresser le paysage
actuel des techniques de débogage à reculons. La section suivante
explique ensuite l'approche choisie ainsi que les détails
d'implémentation de notre outil. Nous présentons ensuite \emph{PEPDB},
le résultat de notre projet de session accompagné d'une section sur les
possibles améliorations au projet.

\section{Objectifs}\label{objectifs}

L'objectif du projet est de développer un débogueur du langage
assembleur PEP/8. Le débogueur aura la capacité d'exécuter le programme
débogué à sens inverse, ce qui permettra au programmeur de retracer la
provenance des bogues a posteriori. Le débogueur supportera le débogage
de code mutant.

Puisque nous ne partons pas d'un projet existant, nous allons devoir
commencer par implémenter un assembleur et un désassembleur. De la
sorte, un utilisateur pourra fournir un fichier source ou un fichier
binaire à l'interpréteur.

Pour réaliser le débogueur, il est impératif de construire un
environnement d'exécution de code machine PEP/8 qui offre la flexibilité
nécessaire pour y ajouter les fonctionnalités de débogage mentionnées
ci-haut. L'interpréteur va effectuer une capture d'informations,
lesquelles seront utilisées par le débogueur afin d'effectuer
l'exécution inverse du programme.

Le débogueur fournira les fonctionnalités normales d'un débogueur soit
les points d'arrêts (breakpoints) et l'exécution pas-à-pas. Il permettra
d'utiliser ces mêmes fonctionnalités en mode d'exécution inverse. Pour
implémenter cette dernière fonctionnalité, nous allons utiliser une
technique de capture d'états se basant sur le maintien des informations
permettant la restitution de l'état précédent l'exécution de chaque
instruction. Cette implémentation supportera le code automodifiant.

\section{Revue de littérature et état de
l'art}\label{revue-de-littuxe9rature-et-uxe9tat-de-lart}

En effectuant notre revue de littérature sur le sujet, nous avons
d'abord remarqué qu'il s'agit d'un sujet qui intéresse les chercheurs en
informatiques depuis des décennies. Nous présentons les trois articles
qui présentent les trois techniques les plus utilisées afin d'effectuer
du débogage à reculons.

Dans son article Efficient Algorithms for Bidirectional Debugging
(2000)\cite{Boothe-2000}, Boothe présente une approche qui part du constat qu'un appel au
débogueur via un trap coûte environ un million de cycles processeur. Il
cherche donc à éviter le plus possible d'effectuer des interruptions. Au
lieu d'injecter des interruptions dans le programme invité, Boothe
ajoute des appels à des fonctions qui vont s'occuper de mettre à jour
des compteurs qui vont permettre d'arrêter l'exécution du programme
exactement là où l'utilisateur le désire. Cette technique est
particulièrement efficace lorsqu'un utilisateur voudra arrêter le
programme après un certain nombre d'itérations dans une boucle ou bien à
une certaine profondeur de récursion, car les débogueurs font ce travail
en effectuant des appels au débogueur pour chaque itération dans la
boucle. Cette importante économie en cycle de processeur permet
d'effectuer le débogage à reculons en réexécutant le programme, parfois
même en plus d'une passe, pour arrêter l'exécution là où l'utilisateur
l'a demandé. Afin de s'assurer du déterminisme de l'exécution, des
captures doivent être effectuées dans le cas de certains appels
systèmes. L'approche de Boothe se base donc sur la réexécution du
programme afin de revenir à un état précédent de l'exécution du
programme.

Akgul propose une autre approche dans son article Instruction-level
Execution for Debugging (2002)\cite{Akgul-2002} qu'il améliore par la suite par dynamic
slicing dans son article suivant A Fast Assembly Level Reverse Execution
Method via Dynamic Slicing (2004). Au lieu d'exécuter le programme de
nouveau, il construit le programme inverse instruction par instruction.
De cette façon, il ne faut pas garder un historique des états
d'exécution du programme. Il faut uniquement pouvoir passer du programme
d'origine au programme inverse. Bien entendu, il doit y avoir capture
d'états dans le cas où des appels non déterministes sont effectués, mais
c'est le cas de toutes les approches que nous avons étudié.

Le troisième article est An Efficient and Generic Reversible Debugger
using the Virtual Machine based Approach (2005) par Koju et al\cite{Koju-2005}. La
technique qu'ils emploient se base sur deux modes d'exécution afin de ne
pas occasionner de surcoût en performance pour un utilisateur qui veut
exécuter le programme sans effectuer de débogage. Celui-ci peut changer
vers un mode débogage lorsqu'il le souhaite. Ce qui nous intéresse
particulièrement dans cet article est la capture d'états effectués à
intervalle de temps dynamiquement ajusté selon une mesure de la durée
requise pour la capture d'état et le surcoût toléré. Pour effectuer
l'exécution inverse, le débogueur retourne au dernier état capturé et
réexécute le code jusqu'au point souhaité.

\section{Implémentation}\label{impluxe9mentation}

\subsection{Suite d'outils
préliminaires}\label{suite-doutils-pruxe9liminaires}

Nous avons implémenté les différents outils nécessaires à
l'implémentation de notre débogueur en Nit. Nous avons développé chaque
outil de façon isolée afin qu'il soit possible d'utiliser chacun
indépendamment. Notre suite d'outils comprend donc un assembleur et un
désassembleur de \emph{PEP/8} et un interpréteur naïf. Nous avons décidé
d'implémenter l'interpréteur naïvement puisque nous voulions être en
mesure de déboguer le comportement exact du code assembleur écrit et pas
une version optimisée. Aussi, une implémentation simple de
l'interpréteur facilite beaucoup l'intégration des mécanismes
d'historique pour le débogage à reculons.

\subsection{Débogueur}\label{duxe9bogueur}

Nous avons opté d'implémenter notre débogueur avec une approche de
machine virtuelle. C'est à dire, d'introduire les mécanismes qui
permettent le débogage du programme à l'intérieur même de son
environnement d'exécution. Ceci nous permet de plus facilement
intercepter les implémentations des différentes instructions assembleurs
afin d'exécuter les routines de vérifications du débogueur. Autrement,
nous aurions dût attacher notre débogueur au processus en cours
d'exécution et d'y injecter des instructions binaires pour arriver au
même résultat.

Les mécanismes que nous avons implémentés dans le débogueur sont les
suivants: la gestion de points d'arrêt et la reprise d'exécution,
l'exécution pas-à-pas, l'inspection de registres et de l'état de la
mémoire ainsi que le désassemblage dynamique des octets en mémoire.
Fonctionnalité d'exécution à reculons Nous avons choisi d'implémenter le
débogage à reculons en sauvegardant l'effet de chaque instruction
exécuté dans un historique d'exécution. En assembleur \emph{PEP/8},
chaque instruction peut modifier au maximum un mot de deux octets en
mémoire. Autrement, l'effet de l'instruction peut être de modifier le
contenu d'un ou plusieurs registres. Nous avons donc implémenté un
interpréteur spécialisé qui effectue le travail additionnel pour
permettre l'exécution à reculons et qui supporte les points d'arrêts.
Pour chaque instruction, nous sauvegardons donc le banc de registre,
l'adresse mémoire affectée s'il y a lieu et la valeur qui s'y trouve. Le
surcoût en mémoire de notre approche est donc tout à fait acceptable
compte tenu des bénéfices de l'exécution à reculons.

Certaines instructions ont un effet non déterministe. En langage
d'assemblage \emph{PEP/8}, ces instructions sont les instructions de
lecture de données usagers: \texttt{CHARI} et \texttt{DECI}. Afin de
maintenir l'application dans le même chemin d'exécution, si après avoir
enregistré une instruction non déterministe, l'usager demande de la
rejouer en sens normal, nous rappliquons la lecture effectuée lors de la
dernière exécution de l'instruction. Cette logique nous permet de
déboguer le chemin d'exécution sans avoir à répéter chaque entrée à
chaque passage.

\subsection{Séparation des tâches}\label{suxe9paration-des-tuxe2ches}

Lors des phases d'implémentation, le travail à accomplir fût divisé en
deux afin d'implémenter plus rapidement les différents composants du
projet. Initialement, Frédéric Vachon implémenta le désassembleur et
Philippe Pépos Petitclerc l'assembleur de code assembleur \emph{PEP/8}.
Ensuite, le travail sur l'interpréteur fût divisé en instructions.
Frédéric Vachon a ensuite codé la majeure partie du débogueur initial.
Auquel les deux membres du projet ont ultérieurement contribué lorsque
nécessaires. La logique d'enregistrement d'état a d'abord été écrite par
Philippe Pépos Petitclerc et retravaillée par Frédéric Vachon pour
faciliter la ré-exécution d'instructions déjà enregistrées et gérer le
cas des entrées usagers.

\section{Présentation de l'outil}\label{pruxe9sentation-de-loutil}

Nous présentons donc \emph{PEPDB} un débogueur pour le langage
d'assemblage \emph{PEP/8} qui est basé sur une machine virtuelle
sauvegardant les effets des instructions exécutés afin de permettre
l'exécution inverse des instructions. L'outil consiste en une série
d'outils écris en \emph{Nit} qui permettent individuellement
d'assembler, de désassembler, d'interpréter de de déboguer l'exécution
d'un programme \emph{PEP/8}.

Le débogueur (outil principal de la suite) supporte la correspondance
entre le code source et le code octet compilé. Il est donc en mesure
d'afficher les commentaires et les étiquettes même s'ils disparaissent à
l'assemblage.

L'approche utilisée pour l'implémentation du débogueur et de la gestion
de l'historique d'exécution permet facilement de supporter le code auto
modifiant. Par contre, s'il y a réécriture du code original, le
débogueur doit invalider la correspondance avec les fichiers sources.

Une utilisation typique du débogueur consiste en 3 étapes: Le chargement
du code \emph{PEP/8}, l'exécution jusqu'à l'occurrence du bogue
recherché, l'exécution inverse jusqu'à l'instruction qui introduit le
comportement imprévu. Le débogueur charge automatiquement le fichier
source passé en argument en mémoire. Il est donc immédiatement possible
de le désassembler pour voir le source via la commande \texttt{disass}.
La gestion des points d'arrêts se fait via les commandes \texttt{break}
ou \texttt{remove}. Le contrôle de l'exécution est possible via les
commandes \texttt{run}, \texttt{continue} et \texttt{nexti}. Une fois
que le bogue est survenu, l'utilisateur voudra reculer l'exécution via
les versions \texttt{rev-} des mêmes commandes.

Le code des différents outils est trouvable dans le répertoire
\texttt{src} du projet. Le code de l'assembleur et du désassembleur se
trouve respectivement dans les fichiers \texttt{asm.nit} et
\texttt{disasm.nit}. L'interpréteur et la version spécialisée pour le
débogage se trouvent dans \texttt{interpreter.nit} et le débogueur
lui-même dans \texttt{debugger.nit}. Le répertoire \texttt{src} contient
également un fichier de description des instructions \emph{PEP/8} et un
fichier de description des commandes du débogueur. Le fichier
\texttt{Makefile} à la racine du projet compilera les différents outils
et les placera dans le répertoire \texttt{bin} du projet.

\section{Améliorations possibles}\label{amuxe9liorations-possibles}

La première fonctionnalité additionnelle qui serait intéressante est la
possibilité d'avoir un arbre d'exécution. Il serait donc possible de
diverger de l'exécution originale en exécutant une nouvelle version
d'une instruction d'entrée usager. Il serait possible de permettre à
l'usager de choisir la branche d'exécution lorsque le débogueur arrive a
un branchement et même de proposer d'en créer une nouvelle.

Une autre amélioration éventuelle serait de construire une interface
graphique qui n'est pas en ligne de commande. Comme le langage
d'assemblage \emph{PEP/8} est surtout utilisé à des fins d'apprentissage
et, à l'UQÀM, par des étudiants en début de programme, il serait plus
simple pour eux d'avoir accès à une interface graphique.

\section{Conclusion}\label{conclusion}

Suite à une recherche sur les solutions en débogage à reculons, nous
avons choisi que la solution la plus adaptée à nos besoins fût
d'utiliser une machine virtuelle qui permettrait d'enregistrer les
effets de l'exécution d'une instruction afin de permettre de l'exécuter
dans n'importe quel sens par la suite. Nous avons donc implémenté une
suite d'outils permettant de travailler avec le langage \emph{PEP/8} et
avons implémenté un débogueur offrant ces fonctionnalités. Le débogueur
se sert d'un interpréteur spécialisé qui enregistre la différence d'état
de la mémoire ainsi que les registres. L'outil proposé est également en
mesure de faire la correspondance entre le code source original et le
code octet assemblé. Il supporte aussi le débogage de code
automodifiant.

\bibliography{rapport}{}
\bibliographystyle{plain}

\end{document}
